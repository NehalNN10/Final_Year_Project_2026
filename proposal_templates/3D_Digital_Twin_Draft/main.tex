\documentclass{article}

\usepackage{array}
\usepackage{etoolbox}
\usepackage{fancyhdr}
\usepackage{geometry} 
\usepackage{graphicx}
\usepackage{soul}
\usepackage{titling}
\usepackage{natbib} % for references

%%%%%%%%%%%%%%%%%%%%%%%%%%%%%%%%%%%%%%%%%%%%%%%%%%%%%%%%%%%%
% BEGIN METADATA: Edit the following as appropriate
%%%%%%%%%%%%%%%%%%%%%%%%%%%%%%%%%%%%%%%%%%%%%%%%%%%%%%%%%%%%

\title{HU VR Workrooms}  % the title of your project
\newcommand\shorttitle{\thetitle}  % if needed: a shorter title for the document header
% Team members.
\newcommand\firstname{Nehal Naeem Haji}  % full name
\newcommand\firstid{nh07884}         % ID, e.g. xy01234
\newcommand\secondname{Manal Hasan} % full name
\newcommand\secondid{mh08438}        % ID, e.g. xy01234
\newcommand\thirdname{Eeshal Khalidnadeem Qureshi}  % full name
\newcommand\thirdid{id03}         % ID, e.g. xy01234
% Uncomment the rows for the next 2 students if and as needed.
\newcommand\fourthname{Muhammad Shawaiz Khan} % full name
\newcommand\fourthid{mk07899}        % ID, e.g. xy01234
% \newcommand\fifthname{Student 5}  % full name
% \newcommand\fifthid{id05}         % ID, e.g. xy01234

%%%%%%%%%%%%%%%%%%%%%%%%%%%%%%%%%%%%%%%%%%%%%%%%%%%%%%%%%%%%
% END METADATA: Do not edit the preamble any further.
%%%%%%%%%%%%%%%%%%%%%%%%%%%%%%%%%%%%%%%%%%%%%%%%%%%%%%%%%%%%

\pagestyle{fancy}
\lhead{Kaavish Proposal}
\chead{\shorttitle}
\rhead{Fall 2025}
\cfoot{Page \thepage}
\renewcommand{\footrulewidth}{0.4pt}

\newcommand\instruction[1]{\textit{#1}}

\begin{document}

% Cover page.
\input{cover}

%%%%%%%%%%%%%%%%%%%%%%%%%%%%%%%%%%%%%%%%%%%%%%%%%%%%%%%%%%%%
% DATA: Populate the rest of the document as instructed.
%%%%%%%%%%%%%%%%%%%%%%%%%%%%%%%%%%%%%%%%%%%%%%%%%%%%%%%%%%%%
\section{Abstract}
\instruction{Please write a 500-600 word abstract on the project idea. It should not be very technically written but should be understandable by anyone.}


\section{Problem definition}
\instruction{Describe the problem that the project addresses.}\\\\
Students and faculty at Habib University—and across Pakistani higher education—face significant engagement and accessibility challenges with existing remote collaboration tools. Traditional video conferencing platforms induce virtual meeting fatigue in 87\% of participants within 30 - 40 minutes, driven by unnatural eye-contact demands, constant self-view monitoring, and cognitive overload. Premium VR meeting solutions (e.g., Meta Horizon Workrooms) offer immersive collaboration but remain inaccessible to 99\% of Pakistani students, as VR headset adoption in Pakistan is only 1\% and devices cost over \$430 each—equivalent to several months' average income. Conversely, 85\% of Pakistani university students own smartphones, yet these devices are underutilized for immersive collaboration, creating a two-tiered digital divide where economically constrained students are relegated to flat 2D interfaces despite owning capable hardware. This project seeks to democratize immersive remote collaboration by leveraging WebXR and mobile VR to bridge this accessibility gap.

\section{Social relevance}
\instruction{Describe any societal problem that the project addresses.}

\section{Originality/Novelty}
\instruction{Describe the value of solving the problem. Compare and contrast with any existing solutions.}

\section{CS contribution}
\instruction{Describe the CS component of the project, e.g. the higher level CS courses that contribute to it.}

\section{Scope and Deliverables}
\instruction{Justify the scope of the project with respect to the size of the team and the year long duration. List the foreseeable deliverables.}

\section{Feasibility}
\instruction{List the resources, e.g. datasets, compute resources, software libraries, hardware, required for the project. Mention how you expect to access and utilize them for the project.}

\section{Team dynamics}
\instruction{Justify the suitability of the team members to the project. For example, their relevant courses, projects, internships, or research.}

\section{Tech stack}
\instruction{Write details of the tech stack you will use for this project for e.g. if you are using MERN stack, you can write MongoDB, Express, React and NodeJS etc.}

\section{References}
\instruction{List your references.}

% External advisor undertaking.
\input{external}

\end{document}

%%% Local Variables:
%%% mode: latex
%%% TeX-master: t
%%% End:
